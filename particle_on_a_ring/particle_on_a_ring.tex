\documentclass{article}
\begin{document}
\section{Particle on a ring}

The following Python script will plot the real components of wavefunctions for a particle on a ring. \par\medskip 
\noindent The script uses wavefunction from the lecture material:
\[\psi (\phi) = N(\cos m_l \phi \pm i \sin m_l \phi) \]

\section{How to use the script}
The quantum number, $m_l$, should be changed to show different wavefunctions. Running the script by clicking Activate, followed by Run, will produce a 3D plot of the wavefunction's real component.

\section{Things to try}
\begin{enumerate}
\item Read the script and take note of the parameters and formulae. Run the code.
\item Make drawings of what the plots should look like for $m_l=\pm 1$, $\pm 2$, $\pm 3$ and $\pm 4$. Run the script. Were your drawings correct? 
\item EXTENSION: Change the function for \texttt{psi} to: \par\texttt{psi = np.cos(2 * phi) * np.cos(3 * phi)}\par Run the script. What does the plot show? Are the wavefunctions for $m_l = 2$ and $m_l = 3$ orthogonal?
\end{enumerate}

\end{document}